\chapter{Architektur}

    In folgendem Kapitel soll die Architektur des Softwaresystems beschrieben
    werden.

    Die Software besteht aus mehreren Teilmodulen, welche in
    \autoref{img:packages} beschrieben ist.
    Jedes dieser Teilmodule wurde unabhängig voneinander entwickelt. In einem
    abschließenden Entwicklungsschritt wurdden die Teilmodule zusammengefügt.

    \section{Module}

        In folgendem soll kurz auf die einzelnen Module eingegangen werden.

        \subsection{Behaviour}

            Dieses Modul beinhaltet den Quellcode welcher zum Aufrufen der
            Behaviours auf dem Roboter benötigt werden.
            Es beinhaltet ein Untermodul ``Nao``, welches funktionalität
            beinhaltet welche Nao behaviours aufrufen kann welche keine
            speziellen Eingabewerte benötigen.
            Ein Beispiel dafür ist die Begrüßung oder Verabschiedung am Anfang
            bzw. Ende des Spiels.
            Dieses Behaviour ist immer gleich und bedarf keiner speziellen
            Eingabeparameter.

            Das Modul exportiert Funktionen welche in
            \autoref{img:package:behaviour} aufgezeigt werden.

        \subsection{Algorithmus}

            Das Modul welches den Algorithmus beinhaltet Funktionalitäten welche
            den Spielealgorithmus abbilden.

            \autoref{img:algo:classes} zeigt die Funktionalitäten in UML auf.

        \subsection{OpenCV}

