\chapter{Spiele-Algorithmus}

    Der Algorithmus welcher den Ablauf des Spiels beschreibt ist in
    \ref{img:algo:sequence} beschrieben.

    Das Softwaremodul welches den Algorithmus implementiert hat dabei eine
    wohldefinierte Schnittstelle, welche nach aussen verfügbar ist.
    Ein Benutzer der Schnittstelle hat verschiedene Usecases, welche in
    \ref{img:algo:usecases} visualisiert sind.
    Der Ablauf des Spieles ist zudem in Abbildung \ref{img:algo:sequence}
    umrissen.

    Das resultierende Klassendiagramm (\ref{img:algo:classes}) zeigt die
    Algorithmus-Komponente.
    Es existieren drei Helfer-Klassen, welche in Folgendem beschrieben werden
    sollen.

    \section{Helferklasse: Card}

        Diese Klasse repräsentiert eine ``Karte'' im Spiel.
        Sie existiert um eine Karte zu abstrahieren und stellt ein einfaches
        Interface für den Algorithmus dar, mit Karten zu hantieren.

    \section{Helferklasse: State}

        Diese Klasse (eine Enumeration) stellt den aktuellen Status des
        Algorithmus dar.
        Der Algorithmus kann in verschiedenen Zuständen sein, wie zum Beispiel
        ``Spielend'' oder ``Angehalten''.

    \section{Helferklasse: Draw}

        Diese Klasse (eine Enumeration) stellt die Antwort auf die Frage, ob
        noch einmal gezogen werden soll, dar.
        Ein einfacher Wahrheitswert wäre zwar ausreichend, beschreibt allerdings
        nur ob gezogen werden soll oder nicht.
        Diese Klasse bietet eine abstraktere Antwort, damit der Roboter
        entsprechend verdeutlichen kann, wie sicher er sich ist beim nächsten
        Zug.

% vim: spelllang=de
