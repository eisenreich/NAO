\chapter{Installation}

    Folgendes Kapitel beschreibt die Installation der Entwicklungsumgebung mitsamt
    Toolchain und ``qibuild''.

    \section{Ubuntu 14.04}

        In folgendem Kapitel wird eine virtuelle Maschine mit Ubuntu
        verwendet.
        Es wurde die Ubuntuversion aus Listing \ref{lst:ubuntuvers} verwendet.

        \defaultlst{src/sources/ubuntu-os-release.sh}
                    {bash}
                    {Ubuntu version}
                    {lst:ubuntuvers}

        Im Ersten Schritt werden alle benötigten Pakete installiert (siehe
        Listing \ref{lst:aptget}).
        Dazu wird das Systemtool ``apt-get'' verwendet.

        \defaultlst{src/sources/installation-packages.sh}
                    {bash}
                    {Installation der Pakete}
                    {lst:aptget}
        (entnommen aus \cite[S. 5, f.]{projss15})

        Zur Verwendung der Toolchain wird im Folgenden eine Ordnerstruktur
        angelegt, in welchem der Quellcode verwaltet wird (Listing
        \ref{lst:mkdirp}).

        \defaultlst{src/sources/folder-structure.sh}
                    {bash}
                    {Anlegen der Ordnerstruktur}
                    {lst:mkdirp}
        (entnommen aus \cite[S. 4]{projss15})

        Die Programmierumgebung ``qibuild'' welche die Toolchain zur Verfügung
        stellt muss installiert und konfiguriert werden (Listing
        \ref{lst:installqibuild}).

        \defaultlst{src/sources/install-qibuild.sh}
                    {bash}
                    {Installieren von qibuild}
                    {lst:installqibuild}
        (entnommen aus \cite[S. 5]{projss15})

        Nachdem die Umgebung erfolgreich installiert wurde, kann eine Toolchain
        angelegt werden (Listing \ref{lst:mktoolchain} und ``qibuild''
        initialisiert werden (Listing \ref{lst:initqibuild}).

        \defaultlst{src/sources/mk-toolchain.sh}
                    {bash}
                    {Erstellen der Toolchain}
                    {lst:mktoolchain}
        (entnommen aus \cite[S. 7]{projss15})

        \defaultlst{src/sources/qibuild-init.sh}
                    {bash}
                    {Init: qibuild}
                    {lst:initqibuild}
        (entnommen aus \cite[S. 8, f.]{projss15})

        Die Konfigurationsdatei für ``qibuild''
        (im Pfad ~/.config/qi/qibuild.xml) muss bearbeitet werden, die
        Toolchain muss eingetragen werden (Listing \ref{lst:qibuildxml}).

        \defaultlst{src/sources/qibuild.xml}
                    {xml}
                    {qibuild XML file}
                    {lst:qibuildxml}
        (entnommen aus \cite[S. 8]{projss15})

